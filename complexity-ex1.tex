% !TeX spellcheck = en_US
\documentclass[english]{uebung_cs}
\usepackage{complexity}
\uebung{1}{}{}
\blattname{Mandatory Assignment 1}
%%%%%%%%%%%%%%%%%%%%%%%%%%%%%%%%%%%%%%%%%%%%%%%%%%%%%%%%%%%%%%%%%%%%%%%%%%%%
\begin{document}

\begin{itemize}
  \item Hand-in deadline for the written solutions: Tuesday, May 24, at 10:15 in H9.
  \item You are encouraged to work in teams of size $\le 2$ and hand in your solutions together.
  \item We expect proofs to be formal, detailed, and complete.
\end{itemize}

\begin{aufgabe}[Diagonalization]\mbox{}\\
    Let $f,g\colon\N\to\N$ be space-constructible functions with $f(n)\le o(g(n))$.
    \textbf{Formally prove} that $\mathbf{SPACE}(f(n))\subsetneq\mathbf{SPACE}(g(n))$ holds.

    \textit{Hint: Use diagonalization to construct a language~$L\in\mathbf{SPACE}(g(n))$ with $L\not\in\mathbf{SPACE}(f(n))$.}

    \textit{You can use the following fact without proving it:}
    There exists a deterministic Turing machine $\mathcal{SU}$ such that, for every string~$\alpha$ and input~$x$, if the Turing machine $M_\alpha$ represented by $\alpha$ halts on $x$, then
      $\mathcal{SU}(\alpha,x)=M_\alpha(x)$ holds and
      $\mathcal{SU}$ only ever uses at most $C_\alpha\cdot s_\alpha(x)$ cells of its work tapes, where $C_\alpha$ is a constant only depending on $\alpha$ and $s_\alpha(x)$ is the number of memory cells used by $M_\alpha$ on input~$x$.
\end{aufgabe}

\begin{aufgabe}[Oracles]\mbox{}\\
  Let $M$ be a deterministic oracle Turing machine that only queries strings 
  that are shorter than the input string.
  Let $A,B\subseteq\{0,1\}^\ast$ be two languages.
  Formally prove the following: If $A = L(M^A)$ and $B = L(M^B)$, then $A = B$.
  
  \emph{Hint:} Prove by induction over $n$ that $A^{\le n} = B^{\le n}$ holds. Here $L^{\le n}=\{x\in L\colon |x|\le n\}$.
\end{aufgabe}

\begin{aufgabe}[Integer Factorization]\mbox{}\\
  Let $\mathtt{FACTOR}=\{\langle\operatorname{bin}(x),\operatorname{bin}(y)\rangle\colon \text{$x,y\in\N$ and there is an integer $i$ with $2\le i \le y$ and $i \mathrel{|} x$}\}$,
  where $\operatorname{bin}(x)$ denotes the binary encoding of the non-negative integer~$x$.
  Prove the following.
  \begin{enumerate}
    \item $\mathtt{FACTOR}\in\mathbf{NP}$.
    \item $\mathtt{FACTOR}\in\mathbf{coNP}$. (As a black box, you can use the result by Agrawal, Kayal, and Saxena that $\mathtt{PRIMES}\in\mathbf{P}$, that is, there is a polynomial-time algorithm to decide whether $\operatorname{bin}(q)$ is the binary encoding of a prime number~$q$.)
  \end{enumerate}
\end{aufgabe}
\end{document}
