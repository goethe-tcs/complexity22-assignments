% !TeX spellcheck = en_US
\documentclass[english]{uebung_cs}
\usepackage{complexity}
\uebung{3}{}{}
\blattname{Mandatory Assignment 3}
%%%%%%%%%%%%%%%%%%%%%%%%%%%%%%%%%%%%%%%%%%%%%%%%%%%%%%%%%%%%%%%%%%%%%%%%%%%%
\begin{document}

\begin{itemize}
  \item Hand-in deadline for the written solutions: Friday, July 8, at 12:00 in room 301 (Anselm's office) or via E-Mail.
  \item You are encouraged to work in teams of size $\le 2$ and hand in your solutions together.
  \item We expect proofs to be formal, detailed, complete, and concise.
\end{itemize}

\begin{aufgabe}
  Identify the error in the clause gadget (Figure 17.3) and fix it. (Just draw the correct gadget and explain the error in a few bullet points.)
\end{aufgabe}

\begin{aufgabe}
  Show that $\mathrm{PCP}[\Oh(\log n),2] = \mathrm{P}$ holds.
\end{aufgabe}

\begin{aufgabe}
  Show that for every adaptive $(r,q)$-PCP-verifier, there is a nonadaptive $(r,2^q)$-PCP-verifier that verifies the same language.
\end{aufgabe}

\begin{aufgabe}
  A poly-time uniform sampler for satisfiability is a poly-time probabilistic Turing machine
  that outputs one of the satisfying assignments for the input formula
  $\varphi$, uniformly at random. The algorithm may output $\bot$ with
  probability at most $1/4$. If the formula is unsatisfiable, it
  outputs $\bot$ on all random choices.

  Show that if there is a poly-time algorithm for $\#\mathrm{SAT}$, then there is a poly-time uniform sampler for $\mathrm{SAT}$. (You may assume that there is an algorithm that randomly outputs a $0$ or $1$ with the given rational probability value).
\end{aufgabe}

\begin{aufgabe}[Exercise 17.6]
Show that, for every language in AC$^0$, there is a depth three circuit of $n \mathrm{poly}(\log n)$ size that decides it on a $1−1/\mathrm{poly}(n)$ fraction of inputs and looks as follows:
It has a single $\oplus$-gate of unbounded fan-in at the top, and all other gates are $\wedge$-gates of $\vee$-gates of fan-in at most $\mathrm{poly}(\log n)$.

\emph{Hint: Read and use the proof of Lemma 17.17.}
\end{aufgabe}

\end{document}
